\documentclass[margin,line]{res}
\renewcommand\refname{}
% \usepackage{natbib}
%\usepackage[style=authoryear,backend=biber]{biblatex}
\usepackage[backend=biber,style=phys,sorting=none]{biblatex}
\DeclareSourcemap{
 \maps[datatype=bibtex,overwrite=true]{
  \map{
    \step[fieldsource=Collaboration, final=true]
    \step[fieldset=usera, origfieldval, final=true]
  }
 }
}

\renewbibmacro*{author}{%
  \iffieldundef{usera}{%
    \printnames{author}%
  }{%
    \printnames{author} (\printfield{usera})%
  }%
}%
%\usepackage[utf8]{inputenc}
%\usepackage[greek,english]{babel}
%\usepackage{CJKutf8}

\DeclareFieldFormat{labelnumberwidth}{}
\setlength{\biblabelsep}{0pt}

\addbibresource{publications.bib}
\addbibresource{proceedings.bib}
\defbibheading{bibliography}{}

\oddsidemargin -.25in
\evensidemargin -.25in
\textwidth=5.5in
\itemsep=0in
\parsep=0in

\newenvironment{list1}{
  \begin{list}{\ding{113}}{%
      \setlength{\itemsep}{0in}
      \setlength{\parsep}{0in} \setlength{\parskip}{0in}
      \setlength{\topsep}{0in} \setlength{\partopsep}{0in} 
      \setlength{\leftmargin}{0.17in}}}{\end{list}}
\newenvironment{list2}{
  \begin{list}{$\bullet$}{%
      \setlength{\itemsep}{0in}
      \setlength{\parsep}{0in} \setlength{\parskip}{0in}
      \setlength{\topsep}{0in} \setlength{\partopsep}{0in} 
      \setlength{\leftmargin}{0.2in}}}{\end{list}}


\makeatletter
%\renewenvironment{thebibliography}[1]
%     {%\section*{\refname}% <--- outcommented
%      \@mkboth{\MakeUppercase\refname}{\MakeUppercase\refname}%
%      \list{\@biblabel{\@arabic\c@enumiv}}%
%           {\settowidth\labelwidth{\@biblabel{#1}}%
%            \leftmargin\labelwidth
%            \advance\leftmargin\labelsep
%            \@openbib@code
%            \usecounter{enumiv}%
%            \let\p@enumiv\@empty
%            \renewcommand\theenumiv{\@arabic\c@enumiv}}%
%      \sloppy
%      \clubpenalty4000
%      \@clubpenalty \clubpenalty
%      \widowpenalty4000%
%      \sfcode`\.\@m}
%     {\def\@noitemerr
%       {\@latex@warning{Empty `thebibliography' environment}}%
%      \endlist}
%\makeatother

\begin{document}

\name{\Large Dr. Dominic B. Brailsford \vspace*{.1in}}

\begin{resume}
\section{\sc Contact Information}
\vspace{.05in}
\begin{tabular}{@{}p{2in}p{4in}}
Physics Department           & { Phone:}  (+44) 7427546506\\
Bailrigg   & { Email:}    d.brailsford@lancaster.ac.uk \\      
Lancaster  LA1 4YW & \\   
\end{tabular}


\section{\sc Research Interests}
\begin{list2}
\item Neutrino physics: CP violation, neutrino mass, neutrino-nucleus interactions.
\item Reconstruction techniques: pattern recognition, tracking techniques, calorimetry.
\item Time projection chamber physics: electron diffusion, electron lifetime. 
\end{list2}

\section{\sc Education}
{\bf Imperial College London}, London UK\\
\vspace*{-.1in}
\begin{list1}
\item[] Ph.D. in High Energy Physics, March 2016, studying under Dr. Yoshi Uchida.
\end{list1}

{\bf Durham University}, Durham UK\\
\vspace*{-.1in}
\begin{list1}
\item[] MPhys in Physics (First Class), July 2011.
\end{list1}


\section{\sc Technical skills} 
\begin{list2}
\item Analysis Software: ROOT, ART, LArSoft
\item Languages: C++, Python, Pascal
\item Applications:  \LaTeX
\item Operating Systems:  OS/X, Unix/Linux, Windows, Android
\end{list2}



\section{\sc Academic and Work Experience}
{\bf Imperial College London}, London UK

\vspace{-.3cm}
{\em Ph.D Student} --- {\bf October 2011 to April 2015}%\\ 
%\newline
\begin{itemize}
\item Selection of charged-current neutrino interactions in the T2K near detector's Electromagnetic Calorimeter (ECal).
\begin{itemize}
\item Redesign of the ECal reconstruction algorithms from scratch.
\item Development of new pattern recognition algorithms, enhancing the physical capability of the ECal.
\item Assessment of new ECal-based systematic uncertainties, which have now been used in other near detector analyses.
\item The developed selection can now be used in the flagship T2K analyses, enhancing the experiment's sensitivity to neutrino oscillations.
\end{itemize}
\item Magnetic field simulation in the T2K near detector magnetic flux return yoke.
\begin{itemize}
\item Developed the first ever simulation of the magnetic field in the near detector's flux return yoke.
\item Implementation of this simulation massively reduced disagreements between collected data and Monte Carlo simulation in the near detector.
\item The magnetic field simulation is now a permanent feature of the near detector simulation.
\end{itemize}
\item Near detector ECal reconstruction (ecalRecon) and analysis (oaAnalysis) software package manager.
\begin{itemize}
\item Coordinated development of the packages with the numerous software developers.
\item Managed regular tagging/freezes of the software packages.
\item One of the key liaisons during preparation for the largest data and simulation production ever run by the T2K near detector group.
\end{itemize}
\end{itemize}

{\bf Lancaster University}, Lancashire UK

\vspace{-.3cm}
{\em Research Associate} --- {\bf April 2015 to present}%\\ 
\begin{itemize}
\item The DUNE 35~ton Liquid Argon Time Projection Chamber (LArTPC) prototype.
\begin{itemize}
\item Development of simulation filtering software, increasing the number of physics study opportunities for other collaborators.
\item Development of the data acquisition software, overhauling the input configuration structure, which improved efficiency and ease of use.
\item Calibration of the cosmic ray paddles, which were used in every 35~ton analysis.
\item Author of two 35~ton detector papers, which are currently in preparation.
\end{itemize}
\item The DUNE muon neutrino disappearance selection in the far detector.
\begin{itemize}
\item Lead analyser of one of two flagship selection measurements in the DUNE experiment.
%\item Analysis work has been started from scratch.
\item Results of the analysis will be included in the DUNE experiment's technical design report.
\end{itemize}
\item Rock-induced neutrino backgrounds for the DUNE near detector.
\begin{itemize}
\item Successful development of new simulation from scratch, which simulates background particles that enter the DUNE near detector cavern from the surrounding rock.
\item Work originally undertaken as part of the DUNE near detector task force as an invited member.
\item The work scope has been expanded at the request of the DUNE near detector physics convenor.
\end{itemize}
\item SBND Monte Carlo production.
\begin{itemize}
\item Group coordinator.
\item Group tasked with producing massive volumes of data,  which is critical for progression of SBND.
\item Personally built the group from scratch; developing the necessary infrastructure for the group to function.
\item Led the group through multiple critical Monte Carlo productions, disseminating the data to analysers.
\end{itemize}
\item SBND UK Anode Plane Array (APA) cold testing.
\begin{itemize}
\item Group manager.
\item Group tasked with quality assuring the instrumented SBND APAs via cryogenic freezing.
\item Led the group through test design, successful prototyping and production testing.
\end{itemize}
\item The High Pressure Time Projection Chamber (HPTPC) at CERN experiment.
\begin{itemize}
\item Lead analyser of the HPTPC experiment's flagship analysis, the results of which will be key inputs to neutrino interaction generators.
\item Manager of the successful beam time proposal to the CERN SPSC. 
\item Manager of the HPTPC physics simulation software, tasked with coordinating the developers whilst also actively developing the software.
\end{itemize}
\end{itemize}



%Research under the supervision of Dr. Yoshi Uchida and Dr. Morgan Wascko, focusing on the selection of charged-current neutrino interactions on lead in the T2K near detector ND280.  The search makes use of the ND280 Electromagnetic Calorimeters (ECals) to make the world's first charged-current cross-section on lead below 1 GeV.  The analysis is based on a set of new reconstruction algorithms which I have developed.

%Other work focused on general software improvements and maintenance of the ECal reconstruction software and the near detector analysis software.  I investigated the effect of the magnetic field simulation in the ND280 flux return yoke on the event rate in the barrel ECals.  This addition to the simulation significantly improved the agreement between simulated and real data in the ECals.  I was one of the driving forces for preparing the ECal software for the most recent ND280 software iteration.  This involved a large amount of software development as well as being the liaison with other detector software groups.

 
% \section{\sc Honors and Awards} 
%\begin{list2}
%\item Martin Deutsch Student Award for Excellence in Experimental Physics, 2008
%\item National Defense Science and Engineering Fellow, 2006-2009
%\item Grainger Senior Scholarship in Physics, 2005-2006
%\item Phi Beta Kappa, 2005
%\item Student Marshal, University of Chicago, 2005
%\item Barry M. Goldwater Scholarship, 2004 
%\end{list2}

 




\section{\sc Collaboration Memberships} 
\begin{list2}
\item T2K (2011-present)
\item DUNE (2015-present)
\item SBND (2015-present)
\item HPTPC (2016-present)
\end{list2}

\section{\sc Collaboration Committees and Responsibilities} 
\begin{list2}
\item 2012-2014 T2K ecalRecon software package manager
\item 2012-2014 T2K oaAnalysis software package manager
\item 2012-2014 T2K nd280AnalysisTools software package manager
\item 2013 member of the T2K ND280 file size reduction task force
\item 2014 chair of the T2K Imperial group meetings
\item 2015-present chair of the Lancaster liquid argon group meetings
\item 2015-present SBND UK APA cold testing group manager
\item 2016-2017 member of the DUNE near detector task force
\item 2016-present SBND Monte Carlo production coordinator
\item 2016-present SBND sbndutil software manager
\item 2017 Manager of the HPTPC SPSC beam time request proposal
\item 2017-present chair of the Lancaster neutrino group meetings
\item 2017-present SBND speakers committee 
\end{list2}

\section{\sc Workshop and conference organisation}
{\bf T2K ECal-As-Target workshop} --- sole host and organiser
\begin{list2}
\item Bailrigg, Lancaster University, Lancashire, UK, 2015
\end{list2}
{\bf Exotic neutrinos workshop} --- host and organiser
\begin{list2}
\item Bailrigg, Lancaster University, Lancashire, UK, 2016
\end{list2}
{\bf Short Baseline Analysis workshop} --- planning committee
\begin{list2}
\item Fermi National Accelerator Laboratory, Illinois, USA, 2017
\end{list2}

\section{\sc Conference Presentations}

{\bf Overview of the T2K Experiment}
\begin{list2}
\item RAL HEP Summer School, Somerville College, Oxford, UK, 2012
\end{list2}
{\bf Measuring charged current neutrino interactions in the T2K near detector ECals}
\begin{list2}
\item IOP Joint High Energy Particle Physics and Astro Particle Physics Groups Annual Meeting, Royal Holloway, University of London, London, UK, 2014
\end{list2}
{\bf Physics Program of the Short-Baseline Near Detector}
\begin{list2}
\item The XXVII International Conference on Neutrino Physics and Astrophysics (Neutrino 2016), Imperial College London, London, UK, 2016
\end{list2}
{\bf Selection of charged-current muon-neutrino and electron-neutrino interactions in the DUNE far detector}
\begin{list2}
\item 2017 Meeting of the APS Division of Particles and Fields (DPF 2017), Fermi National Accelerator Laboratory, Illinois, USA, 2017
\end{list2}
{\bf DUNE: Status and Perspectives}
\begin{list2}
\item Prospects in Neutrino Physics (NuPhys 2017), The Barbican Centre, London, UK, 2017
\end{list2}

\begin{refsection}[proceedings]
\section{\sc Conference Proceedings}
\nocite{*}
\printbibliography
\end{refsection}


\section{\sc Teaching }
\begin{list2}
\item 2012 1st year classes, teaching assistant to Prof. Zulfikar Najmudin
\item 2013 1st year Electricity and Magnetism, teaching assistant to Dr. Daniel Eakins and Dr. Piero Posocco
\item 2013 1st year Special Relativity, teaching assistant to Prof. Jordan Nash and Dr. John Tisch
\item 2013 1st year collections exam marker
\item 2013 1st year Vectors, teaching assistant to Dr. Yvonne Unruh
\item 2014 1st year Vibrations and Waves, teaching assistant to Dr. Yvonne Unruh
\item 2015 UK LArSoft workshop, invited tutor and lecturer
\item 2017 UK LArSoft workshop, invited tutor and lecturer
\end{list2}

\section{\sc Mentoring }
\begin{list2}
\item Supervised Mr. Matthew Garman (Intern)
\item Supervised Miss Jean Morris (Intern)
\item Supervised Miss Alice Baxter (Intern)
\item Supervised Miss Ellen Sandford (Intern)
\item Supervised Mr. Xiao Zhang (Intern)
\item Supervised Mr. Matthew Taylor (Intern)
\item Supervised Miss. Emily Gamblen (Intern)
\item Mentored Mr. Alasdair Knox (PhD candidate)
\item Mentored Mr. Adam Lister (PhD candidate)
\item Mentored Mr. Danny Devitt (PhD candidate)
\item Mentored Dr. Iain Lamont (PhD candidate)
\item Mentored Dr. Luke Pickering (PhD candidate)
\item Mentored Mr. Jasminder Sidhu (MSc candidate)
\item Mentored Mr. Muhsin Ali (MSc candidate)
\item Mentored Miss Tessa Carver (Summer student)
\end{list2}



\newpage
%\section{\sc References}
%\begin{list2}
%\item Dr. Yoshi Uchida\\ Blackett Laboratory 524\\Prince Consort Road\\London\\SW7 2BB\\Phone: +44 (0)20 7594 7821\\Email: y.uchida@imperial.ac.uk\\
%\item Dr. Morgan O. Wascko\\ Blackett Laboratory 525\\Prince Consort Road\\London\\SW7 2BB\\Phone: +44 (0)20 7594 1607\\Email: m.wascko@imperial.ac.uk\\
%\item Dr. Asher C. Kaboth\\ Blackett Laboratory 530\\Prince Consort Road\\London\\SW7 2BB\\Phone: +44 (0)20 7594 7813\\Email: a.kaboth@imperial.ac.uk\\
%\end{list2}

%\section{\sc Hobbies and Interests}
%\begin{list2}
%\item Change ringing of historic tower bells
%\item Independent knitting pattern designer
%\end{list2}


\begin{refsection}[publications]
\nocite{*}
\section{\sc Publications}
\printbibliography
\end{refsection}


\end{resume}
\end{document}




